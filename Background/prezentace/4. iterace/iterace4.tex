\documentclass{beamer}

\usepackage[utf8]{inputenc}
\usepackage[czech]{babel}
\usepackage{times}
\usetheme{PaloAlto}

\setbeamertemplate{frametitle continuation}[from second][]

\begin{document}
\title [NKWP]{Nástroj na kontrolu WWW prezentací}
\author[J.~ Koza, A.~ Kugler, J.~Máca, P.~Svoboda]{Jakub~Koza \and Adam~Kugler \and Jindřich~Máca \and Petr~Svoboda}

%číslo následujicí iterace
\subtitle{4. iterace}
\frame{\titlepage}
\begin{frame}[allowframebreaks]\frametitle{Co jsme dělali}
   \begin{itemize}
   %vyplníme dohromady hlavně podle harmonogramu
    \item Vytváření grafu průchodu pomocí nástroje Graphviz
	\item Zlepšování prezentační vrstvy
	\item Oprava chyb
   \end{itemize}
\end{frame}

\begin{frame}[allowframebreaks]\frametitle{Kdo co dělal} 
  \begin{itemize}
    \item Jakub Koza
      \begin{itemize}
       %vyplní Kozík
       \item napojení na GraphViz a testování aplikace (5.75)
       \item oprava kódování stránek - zjišťování z meta tagů (1.5)
       \item nastavení vzdáleného serveru a testování rozdílů mezi verzemi GraphVizu (3.15)
       \item oprava logování a komentářů, generování API (1.5)
     \end{itemize}

    \item Adam Kugler
      \begin{itemize}
      %vyplní Adam
       \item tvorba a úprava grafových generátorů (6 hodin)
       \item testování grafů (5 hodin)
       \item sepsání možných vylepšení na Wiki (0,5 hodiny)
     \end{itemize}

    \item Jindřich Máca
      \begin{itemize}
      %vyplní Tuník 
       \item Jindrova práce (počet hodin)
     \end{itemize}

    \item Petr Svoboda
      \begin{itemize}
      %vyplní Petr
       \item Petrova práce (počet hodin)
     \end{itemize}
   \end{itemize}  
\end{frame} 

\begin{frame}[allowframebreaks]\frametitle{Důležité fragmenty naší práce}
  \begin{itemize}
    %vylpníme dohromady
    \item Duležitá věc %následuje ukázka
  \end{itemize}
\end{frame}

\begin{frame}[allowframebreaks]\frametitle{Na co jsme hrdí} 
  \begin{itemize}
    \item Jakub Koza
      \begin{itemize}
       %vyplní Kozík
       \item Odolnost aplikace vůči pádu GraphVizu
       \item Opraveno mnoho malých chyb
     \end{itemize}
   
    \item Adam Kugler
      \begin{itemize}
      %vyplní Adam
       \item SVG graf obsahuje všechny informace, které obsahuje TraversalGraph
       \item Pro malé weby je SVG graf velmi dobrý
     \end{itemize}

    \item Jindřich Máca
      \begin{itemize}
      %vyplní Tuník 
       \item Jindrova hrdost
      \end{itemize}  
   
    \item Petr Svoboda
      \begin{itemize}
      %vyplní Petr
       \item Petrova hrdost
     \end{itemize}
   \end{itemize}  
\end{frame}

\begin{frame}[allowframebreaks]\frametitle{Co jak šlo}
  \begin{itemize}
    \item Jakub Koza
     %vyplní Kozík
     \begin{block}{Špatně} %co nešlo
       \begin{itemize}
        \item Odlišnost generování grafů pomocí twopi
       \end{itemize}
     \end{block}
     \begin{block}{Průměrně} %co šlo
        \begin{itemize}
        \item Oprava chyb
       \end{itemize}
     \end{block}
     \begin{block}{Dobře} %co šlo dobře
       \begin{itemize}
        \item Napojení aplikace na volání GraphViz procesu
       \end{itemize}
     \end{block}
   
    \item Adam Kugler
      %vyplní Adam
      \begin{block}{Špatně} %co nešlo
       \begin{itemize}
        \item Testování grafu (časová náročnost velkých grafů)
       \end{itemize}
     \end{block}
     \begin{block}{Průměrně} %co šlo
        \begin{itemize}
        \item Implementace grafových generátorů
       \end{itemize}
     \end{block}
     \begin{block}{Dobře} %co šlo dobře
       \begin{itemize}
        \item sepsání možných vylepšení na Wiki
       \end{itemize}
     \end{block}
  
    \item Jindřich Máca
      %vyplní Tuník 
      \begin{block}{Špatně} %co nešlo
       \begin{itemize}
        \item Těžké věci
       \end{itemize}
     \end{block}
     \begin{block}{Průměrně} %co šlo
        \begin{itemize}
        \item Střední věci
       \end{itemize}
     \end{block}
     \begin{block}{Dobře} %co šlo dobře
       \begin{itemize}
        \item Lehké věci
       \end{itemize}
     \end{block}
   
    \item Petr Svoboda
      %vyplní Petr
      \begin{block}{Špatně} %co nešlo
       \begin{itemize}
        \item Těžké věci
       \end{itemize}
     \end{block}
     \begin{block}{Průměrně} %co šlo
        \begin{itemize}
        \item Střední věci
       \end{itemize}
     \end{block}
     \begin{block}{Dobře} %co šlo dobře
       \begin{itemize}
        \item Lehké věci
       \end{itemize}
     \end{block}
   \end{itemize}
\end{frame}

\begin{frame}[allowframebreaks]\frametitle{Plán do příští iterace}
  \begin{itemize}
    %vylpní vedoucí podle harmonogramu + se mohou věci přidat
    \item Plán
  \end{itemize}
\end{frame}

\begin{frame}[allowframebreaks]\frametitle{Přerozdělení bodů}
    
    \begin{center}
  \begin{tabular}{| c | c | c | c |}
    \hline
     J.~Koza & A.~Kugler & J.~Máca & P.~Svoboda \\
    \hline
    %vylpní vedoucí
     body & body & body & body \\
    \hline
  \end{tabular}     
   \end{center}
\end{frame}

\end{document}