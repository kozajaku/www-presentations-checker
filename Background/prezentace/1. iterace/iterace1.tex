\documentclass{beamer}

\usepackage[utf8]{inputenc}
\usepackage[czech]{babel}
\usepackage{times}
\usetheme{PaloAlto}

\setbeamertemplate{frametitle continuation}[from second][]

\begin{document}
\title [NKWP]{Nástroj na kontrolu WWW prezentací}
\author[J.~ Koza, A.~ Kugler, J.~Máca, P.~Svoboda]{Jakub~Koza \and Adam~Kugler \and Jindřich~Máca \and Petr~Svoboda}

%číslo následujicí iterace
\subtitle{1. iterace}
\frame{\titlepage}
\begin{frame}[allowframebreaks]\frametitle{Co jsme dělali}
   \begin{itemize}
   %vyplníme dohromady hlavně podle harmonogramu
    \item Kernel
		\item Model
		\item Parser
		\item Persistence
		\item Presentaion
		\item Utils
		\item WebCrawler
   \end{itemize}
\end{frame}

\begin{frame}[allowframebreaks]\frametitle{Kdo co dělal} 
  \begin{itemize}
    \item Jakub Koza
      \begin{itemize}
       %vyplní Kozík
       \item Kubova práce (počet hodin)
     \end{itemize}    

    \item Adam Kugler
      \begin{itemize}
      %vyplní Adam
       \item Příprava šablony prezentace (3 hodiny)
       \item Implementace balíčku model.graph (2 hodiny)
       \item Implementace web crawleru (10 hodin)
       \item Implementace HTML parseru (linky, Jsoup) (1,5 hodiny)
       \item Oprava chyb (2 hodiny)
     \end{itemize}

    \item Jindřich Máca
      \begin{itemize}
      %vyplní Tuník 
       \item Implementace Modelu (1 hodinu)
			 \item Implementace Utils (0.25 hodiny)
			 \item Nastylování webového rozhraní (1.5 hodiny)
			 \item Implementace třídy PageReciever v rámci WebCrawleru (6 hodin)
			 \item Refaktoring názvů balíčků (0.5 hodiny)
			 \item Implementace CSS Parseru v rámci Parseru (2 hodiny)
     \end{itemize}

    \item Petr Svoboda
      \begin{itemize}
      %vyplní Petr
       \item Petrova práce (počet hodin)
     \end{itemize}
   \end{itemize}  
\end{frame} 

\begin{frame}[allowframebreaks]\frametitle{Důležité fragmenty naší práce}
  \begin{itemize}
    %vylpníme dohromady
    \item Web Crawler
		\item Prezentační vrstva
		\item Perzistentní vrstva
  \end{itemize}
\end{frame}

\begin{frame}[allowframebreaks]\frametitle{Na co jsme hrdí} 
  \begin{itemize}
    \item Jakub Koza
      \begin{itemize}
       %vyplní Kozík
       \item Kubova hrdost
     \end{itemize}
   
    \item Adam Kugler
      \begin{itemize}
      %vyplní Adam
       \item Web crawler umí tvořit graf již nyní
       \item Náš nástroj začíná vypadat dobře
     \end{itemize}

    \item Jindřich Máca
      \begin{itemize}
      %vyplní Tuník 
       \item I když každý vyvíjel svou část, tak když se to nakonec dalo dohromady a poprvé spustilo, tak to fungovalo
      \end{itemize}  
   
    \item Petr Svoboda
      \begin{itemize}
      %vyplní Petr
       \item Petrova hrdost
     \end{itemize}
   \end{itemize}  
\end{frame}

\begin{frame}[allowframebreaks]\frametitle{Co jak šlo}
  \begin{itemize}
    \item Jakub Koza
     %vyplní Kozík
     \begin{block}{Špatně} %co nešlo
       \begin{itemize}
        \item Těžké věci
       \end{itemize}
     \end{block}
     \begin{block}{Průměrně} %co šlo
        \begin{itemize}
        \item Střední věci
       \end{itemize}
     \end{block}
     \begin{block}{Dobře} %co šlo dobře
       \begin{itemize}
        \item Lehké věci
       \end{itemize}
     \end{block}
   
    \item Adam Kugler
      %vyplní Adam
      \begin{block}{Špatně} %co nešlo
       \begin{itemize}
        \item Web crawler
       \end{itemize}
     \end{block}
     \begin{block}{Průměrně} %co šlo
        \begin{itemize}
        \item Šablona prezentace
       \end{itemize}
     \end{block}
     \begin{block}{Dobře} %co šlo dobře
       \begin{itemize}
        \item HTML parser pomocí Jsoup
        \item Graf v modelu
       \end{itemize}
     \end{block}
  
    \item Jindřich Máca
     %vyplní Tuník 
      \begin{block}{Špatně} %co nešlo
       \begin{itemize}
        \item Page Receiver
       \end{itemize}
     \end{block}
     \begin{block}{Průměrně} %co šlo
        \begin{itemize}
        \item CSS Parser
       \end{itemize}
     \end{block}
     \begin{block}{Dobře} %co šlo dobře
       \begin{itemize}
        \item Nastylování webového rozhraní
       \end{itemize}
     \end{block}
   
    \item Petr Svoboda
      %vyplní Petr
      \begin{block}{Špatně} %co nešlo
       \begin{itemize}
        \item Těžké věci
       \end{itemize}
     \end{block}
     \begin{block}{Průměrně} %co šlo
        \begin{itemize}
        \item Střední věci
       \end{itemize}
     \end{block}
     \begin{block}{Dobře} %co šlo dobře
       \begin{itemize}
        \item Lehké věci
       \end{itemize}
     \end{block}
   \end{itemize}
\end{frame}

\begin{frame}[allowframebreaks]\frametitle{Plán do příští iterace}
  \begin{itemize}
    %vylpní vedoucí podle harmonogramu + se mohou věci přidat
    \item Výpis TraversalGraphu v prezentační vrstvě
		\item Validace HTML pomocí w3c validátoru
		\item Validace CSS pomocí w3c validátoru
		\item Výpis validačních výsledků v prezentační vrstvě
  \end{itemize}
\end{frame}

\begin{frame}[allowframebreaks]\frametitle{Přerozdělení bodů}
    
    \begin{center}
  \begin{tabular}{| c | c | c | c |}
    \hline
     J.~Koza & A.~Kugler & J.~Máca & P.~Svoboda \\
    \hline
    %vylpní vedoucí
     3 & -2 & -2 & 1 \\
    \hline
  \end{tabular}     
   \end{center}
\end{frame}

\end{document}