\documentclass{beamer}

\usepackage[utf8]{inputenc}
\usepackage[czech]{babel}
\usepackage{times}
\usetheme{PaloAlto}

\setbeamertemplate{frametitle continuation}[from second][]

\begin{document}
\title [NKWP]{Nástroj na kontrolu WWW prezentací}
\author[J.~ Koza, A.~ Kugler, J.~Máca, P.~Svoboda]{Jakub~Koza \and Adam~Kugler \and Jindřich~Máca \and Petr~Svoboda}

%číslo následujicí iterace
\subtitle{1. iterace}
\frame{\titlepage}
\begin{frame}[allowframebreaks]\frametitle{Co jsme dělali}
   \begin{itemize}
   %vyplníme dohromady hlavně podle harmonogramu
    \item Kernel
		\item Model
		\item Parser
		\item Persistence
		\item Presentaion
		\item Utils
		\item WebCrawler
   \end{itemize}
\end{frame}

\begin{frame}[allowframebreaks]\frametitle{Kdo co dělal} 
  \begin{itemize}
    \item Jakub Koza
      \begin{itemize}
       %vyplní Kozík
       \item Implementace persistenční vrstvy (10 hodin)
       \item Implementace kernelu (15 hodin)
       \item Konfigurace serveru a modulů (2 hodiny)
     \end{itemize}    

    \item Adam Kugler
      \begin{itemize}
      %vyplní Adam
       \item Příprava šablony prezentace (3 hodiny)
       \item Implementace balíčku model.graph (2 hodiny)
       \item Implementace WebCrawleru (10 hodin)
       \item Implementace HTML Parseru (linky, Jsoup) (1,5 hodiny)
       \item Oprava chyb (2 hodiny)
     \end{itemize}

    \item Jindřich Máca
      \begin{itemize}
      %vyplní Tuník 
       \item Implementace Modelu (1 hodinu)
			 \item Implementace Utils (0.25 hodiny)
			 \item Nastylování webového rozhraní (1.5 hodiny)
			 \item Implementace třídy PageReciever v rámci WebCrawleru (6 hodin)
			 \item Refaktoring názvů balíčků (0.5 hodiny)
			 \item Implementace CSS Parseru v rámci Parseru (2 hodiny)
     \end{itemize}

    \item Petr Svoboda
      \begin{itemize}
      \item Modelování prezentační vrstvy (2.5 hodiny)
      \item Implementace prezentační vrstvy (24,7 hodiny)
     \end{itemize}
   \end{itemize}  
\end{frame} 

\begin{frame}[allowframebreaks]\frametitle{Důležité fragmenty naší práce}
  \begin{itemize}
    %vylpníme dohromady
    \item Web Crawler
		\item Prezentační vrstva
		\item Perzistentní vrstva
  \end{itemize}
\end{frame}

\begin{frame}[allowframebreaks]\frametitle{Na co jsme hrdí} 
  \begin{itemize}
    \item Jakub Koza
      \begin{itemize}
       %vyplní Kozík
				\item Kernel podporuje mnohovláknovost a je schopný regenerace po pádu serveru
				\item Persistenční vrstva je prakticky ve finální podobě
     \end{itemize}
   
    \item Adam Kugler
      \begin{itemize}
      %vyplní Adam
				\item WebCrawler umí tvořit graf již nyní
				\item Náš nástroj začíná vypadat dobře
     \end{itemize}

    \item Jindřich Máca
      \begin{itemize}
      %vyplní Tuník 
        \item PageReceiver si umí poradit i s https protokolem
				\item Náš nástroj začíná vypadat dobře
      \end{itemize}  
   
    \item Petr Svoboda
      \begin{itemize}
      %vyplní Petr
				\item Prezentační vrstva využívá technologii JSF 2
				\item Nástroj disponuje podporou pro lokalizaci
     \end{itemize}
   \end{itemize}  
\end{frame}

\begin{frame}[allowframebreaks]\frametitle{Co jak šlo}
  \begin{itemize}
    \item Jakub Koza
     %vyplní Kozík
     \begin{block}{Špatně} %co nešlo
       \begin{itemize}
        \item Náročné testování kernelu kvůli možným časově závislým chybám
       \end{itemize}
     \end{block}
     \begin{block}{Průměrně} %co šlo
        \begin{itemize}
        \item Persistenční vrstva a db schéma
       \end{itemize}
     \end{block}
     \begin{block}{Dobře} %co šlo dobře
       \begin{itemize}
        \item Konfigurace serveru
        \item Nasazení jednotlivých modulů dohromady
       \end{itemize}
     \end{block}
   
    \item Adam Kugler
      %vyplní Adam
      \begin{block}{Špatně} %co nešlo
       \begin{itemize}
        \item WebCrawler
       \end{itemize}
     \end{block}
     \begin{block}{Průměrně} %co šlo
        \begin{itemize}
        \item Šablona prezentace
       \end{itemize}
     \end{block}
     \begin{block}{Dobře} %co šlo dobře
       \begin{itemize}
        \item HTML parser pomocí Jsoup
        \item Graf v modelu
       \end{itemize}
     \end{block}
  
    \item Jindřich Máca
     %vyplní Tuník 
      \begin{block}{Špatně} %co nešlo
       \begin{itemize}
        \item Page Receiver
       \end{itemize}
     \end{block}
     \begin{block}{Průměrně} %co šlo
        \begin{itemize}
        \item CSS Parser
       \end{itemize}
     \end{block}
     \begin{block}{Dobře} %co šlo dobře
       \begin{itemize}
        \item Nastylování webového rozhraní
       \end{itemize}
     \end{block}
   
    \item Petr Svoboda
      %vyplní Petr
      \begin{block}{Špatně} %co nešlo
       \begin{itemize}
        \item technologie JSF je poměrně rozsáhlá a dokumentace není zrovna nejpřívětivější.
       \end{itemize}
     \end{block}
     \begin{block}{Průměrně} %co šlo
        \begin{itemize}
        \item kódování views
       \end{itemize}
     \end{block}
     \begin{block}{Dobře} %co šlo dobře
       \begin{itemize}
        \item podpora pro lokalizaci
       \end{itemize}
     \end{block}
   \end{itemize}
\end{frame}

\begin{frame}[allowframebreaks]\frametitle{Plán do příští iterace}
  \begin{itemize}
    %vylpní vedoucí podle harmonogramu + se mohou věci přidat
    \item Výpis TraversalGraphu v prezentační vrstvě
		\item Validace HTML pomocí w3c validátoru
		\item Validace CSS pomocí w3c validátoru
		\item Výpis validačních výsledků v prezentační vrstvě
  \end{itemize}
\end{frame}

\begin{frame}[allowframebreaks]\frametitle{Přerozdělení bodů}
    
    \begin{center}
  \begin{tabular}{| c | c | c | c |}
    \hline
     J.~Koza & A.~Kugler & J.~Máca & P.~Svoboda \\
    \hline
    %vylpní vedoucí
     3 & -1 & -3 & 1 \\
    \hline
  \end{tabular}     
   \end{center}
\end{frame}

\end{document}