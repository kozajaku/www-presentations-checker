\documentclass{beamer}

\usepackage[utf8]{inputenc}
\usepackage[czech]{babel}
\usepackage{times}
\usetheme{PaloAlto}

\setbeamertemplate{frametitle continuation}[from second][]

\begin{document}
\title [NKWP]{Nástroj na kontrolu WWW prezentací}
\author[J.~ Koza, A.~ Kugler, J.~Máca, P.~Svoboda]{Jakub~Koza \and Adam~Kugler \and Jindřich~Máca \and Petr~Svoboda}

%číslo následujicí iterace
\subtitle{5. iterace}
\frame{\titlepage}
\begin{frame}[allowframebreaks]\frametitle{Co jsme dělali}
   \begin{itemize}
   %vyplníme dohromady hlavně podle harmonogramu
    \item Průběžné zobrazování výsledků kontroly
		\item Programátorská příručka
		\item Doplnění o konfigurační soubory
		\item Debugging a vylepšování aplikace
		\item Celkové vyhodnocení projektu
   \end{itemize}
\end{frame}

\begin{frame}[allowframebreaks]\frametitle{Kdo co dělal} 
  \begin{itemize}
    \item Jakub Koza
      \begin{itemize}
       %vyplní Kozík
       \item Kubova práce (počet hodin)
     \end{itemize}

    \item Adam Kugler
      \begin{itemize}
      %vyplní Adam
       \item Programátorská příručka (3 hodiny)
       \item Refaktoring + oprava komentářů u grafů (0,5 hodiny)
       \item Doplnění grafů do uživatelské příručky (0,5 hodiny)
     \end{itemize}

    \item Jindřich Máca
      \begin{itemize}
      %vyplní Tuník 
       \item Jindrova práce (počet hodin)
     \end{itemize}

    \item Petr Svoboda
      \begin{itemize}
      %vyplní Petr
       \item Petrova práce (počet hodin)
     \end{itemize}
   \end{itemize}  
\end{frame} 

%\begin{frame}[allowframebreaks]\frametitle{Důležité fragmenty naší práce}
%  \begin{itemize}
    %vylpníme dohromady
%    \item Duležitá věc %následuje ukázka
%  \end{itemize}
%\end{frame}

\begin{frame}[allowframebreaks]\frametitle{Na co jsme hrdí} 
  \begin{itemize}
    \item Jakub Koza
      \begin{itemize}
       %vyplní Kozík
       \item Kubova hrdost
     \end{itemize}
   
    \item Adam Kugler
      \begin{itemize}
      %vyplní Adam
       \item Máme to konečně za sebou
     \end{itemize}

    \item Jindřich Máca
      \begin{itemize}
      %vyplní Tuník 
       \item Jindrova hrdost
      \end{itemize}  
   
    \item Petr Svoboda
      \begin{itemize}
      %vyplní Petr
       \item Petrova hrdost
     \end{itemize}
   \end{itemize}  
\end{frame}

%\begin{frame}[allowframebreaks]\frametitle{Co jak šlo}
%  \begin{itemize}
%    \item Jakub Koza
     %vyplní Kozík
%     \begin{block}{Špatně} %co nešlo
%       \begin{itemize}
%        \item Těžké věci
%       \end{itemize}
%     \end{block}
%     \begin{block}{Průměrně} %co šlo
%        \begin{itemize}
%        \item Střední věci
%       \end{itemize}
%     \end{block}
%     \begin{block}{Dobře} %co šlo dobře
%       \begin{itemize}
%        \item Lehké věci
%       \end{itemize}
%     \end{block}
   
%    \item Adam Kugler
%      %vyplní Adam
%      \begin{block}{Špatně} %co nešlo
%       \begin{itemize}
%        \item Těžké věci
%       \end{itemize}
%     \end{block}
%     \begin{block}{Průměrně} %co šlo
%        \begin{itemize}
%        \item Střední věci
%       \end{itemize}
%     \end{block}
%     \begin{block}{Dobře} %co šlo dobře
%       \begin{itemize}
%        \item Lehké věci
%       \end{itemize}
%     \end{block}
  
%    \item Jindřich Máca
      %vyplní Tuník 
%      \begin{block}{Špatně} %co nešlo
%       \begin{itemize}
%        \item Těžké věci
%       \end{itemize}
%     \end{block}
%     \begin{block}{Průměrně} %co šlo
%        \begin{itemize}
%        \item Střední věci
%       \end{itemize}
%     \end{block}
%     \begin{block}{Dobře} %co šlo dobře
%       \begin{itemize}
%        \item Lehké věci
%       \end{itemize}
%     \end{block}
   
%    \item Petr Svoboda
      %vyplní Petr
%      \begin{block}{Špatně} %co nešlo
%       \begin{itemize}
%        \item Těžké věci
%       \end{itemize}
%     \end{block}
%     \begin{block}{Průměrně} %co šlo
%        \begin{itemize}
%        \item Střední věci
%       \end{itemize}
%     \end{block}
%     \begin{block}{Dobře} %co šlo dobře
%       \begin{itemize}
%        \item Lehké věci
%       \end{itemize}
%     \end{block}
%   \end{itemize}
%\end{frame}

%\begin{frame}[allowframebreaks]\frametitle{Plán do příští iterace}
%  \begin{itemize}
    %vylpní vedoucí podle harmonogramu + se mohou věci přidat
%    \item Plán
%  \end{itemize}
%\end{frame}

\begin{frame}[allowframebreaks]\frametitle{Přerozdělení bodů}
    
    \begin{center}
  \begin{tabular}{| c | c | c | c |}
    \hline
     J.~Koza & A.~Kugler & J.~Máca & P.~Svoboda \\
    \hline
    %vylpní vedoucí
     3 & 1 & -4 & 0 \\
    \hline
  \end{tabular}     
   \end{center}
\end{frame}

\begin{frame}[allowframebreaks]\frametitle{Celkové vyhodnocení projektu}

	Projekt začal minulý semestr jeho analýzou a návrhem. Bohužel, v tu dobu většina z náš teprve objevovala tyto disciplíny a ještě jsme neměli přesnou představu o výsledné aplikaci. Návrh jsme nakonec udělali pro vybrané technologie (Java OSGi Framework) a ujasnili si tak celkovou představu o projektu. Avšak když jsme se pak před začátkem dalšího semestru k projektu vrátili a přehodnotili ho, rozhodli jsme se pro poměrně radikální změnu technologie (Java EE) a tak i změnu návrhu. Zde také padlo rozhodnutí, že se bude jednat o serverovou aplikaci s webovým rozhraním. Zároveň jsme ale nechtěli zahodit všechnu naši práci z prvního semestru a tak jsme se drželi původní koncepce a tím pak vzniklo i pár věcí, které by se dali ve výsledku navrhnout mnohem lépe (viz. přiložené poznámky). Když byl návrh upraven a byly stanoveny již jasné požadavky na aplikaci, pustili jsme se v rámci druhého semestru do implementace. Zde jsme se setkali s další řadou problémů, ale i elegantních řešení (např. CSSBox). Nakonec se nám podařilo dodržet stanovený harmonogram a úspěšně implementovat ony požadavky. Dovolil bych si tedy projekt prohlásit za úspěšný, ne však bezchybný, ale ve výsledku Vám dnes dodáváme funkční zdokumentovanou aplikaci splňující všechny stanovené požadavky a možná i něco navíc.
	
		\begin{block}{Implementované funkce}
			\begin{itemize}
				\item Kontrola platnosti URL linků.
				\item Validace HTML stránek podle W3C.
				\item Validace CSS souborů podle W3C.
				\item Kontrola redundantních CSS stylů.
				\item Vypsání nebo vykreslení grafů průchodu stránkou.
			\end{itemize}
		\end{block}
   
\end{frame}

\begin{frame}[allowframebreaks]\frametitle{Další vývoj projektu}
		
	Další vývoj projektu se v dohledné době neplánuje hlavně proto, že žádný s členů týmu se na něm, po skončení tohoto semestru, neplánuje dále podílet. Avšak už při návrhu se myslelo na jeho rozšiřitelnost a jeho dosavadní průběh je zdokumentován i z programátorského hlediska a vše je také na \href{https://github.com/kozajaku/www-presentations-checker}{\underline{GitHubu}}, takže projekt je možné i nadále rozvíjet a rozšiřovat o další funkčnosti.
		
\end{frame}

\end{document}