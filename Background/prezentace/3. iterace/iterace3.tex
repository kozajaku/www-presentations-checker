\documentclass{beamer}

\usepackage[utf8]{inputenc}
\usepackage[czech]{babel}
\usepackage{times}
\usetheme{PaloAlto}

\setbeamertemplate{frametitle continuation}[from second][]

\begin{document}
\title [NKWP]{Nástroj na kontrolu WWW prezentací}
\author[J.~ Koza, A.~ Kugler, J.~Máca, P.~Svoboda]{Jakub~Koza \and Adam~Kugler \and Jindřich~Máca \and Petr~Svoboda}

%číslo následujicí iterace
\subtitle{3. iterace}
\frame{\titlepage}
\begin{frame}[allowframebreaks]\frametitle{Co jsme dělali}
   \begin{itemize}
   %vyplníme dohromady hlavně podle harmonogramu
		\item WholePresentationController
		\item CSSRedundancyChecker
		\item Parser
		\item Kernel
		\item Presentation
		\item Instalační příručka
   \end{itemize}
\end{frame}

\begin{frame}[allowframebreaks]\frametitle{Kdo co dělal} 
  \begin{itemize}
    \item Jakub Koza
      \begin{itemize}
       %vyplní Kozík
       		 \item Tvorba/oprava komentářů a generování API (5 hodin)
			 \item Kernel (2 hodiny)
			 \item WholePresentationController (4.3 hodiny)
			 \item Oprava chybného merge (2.5 hodiny)
			 \item Instalační příručka (6 hodin)
     \end{itemize}

    \item Adam Kugler
      \begin{itemize}
      %vyplní Adam
       \item Oprava starší dokumentace (1 hodina)
			 \item Úprava EA modelu (1 hodina)
       \item WholePresentationChecker (1 hodina)
			 \item CSSCode a HTMLCode (2 hodiny)
       \item Začátek CSSRedundancyChecker (3 hodiny)
       \item Volba kontroly externích linků (0.5 hodiny)
			 \item Úprava webcrawleru pro potřeby CRC (1.5 hodiny)
			 \item Návrh grafu v Graphvizu (2 hodiny)
       \item Refaktoring - zrušení nutnosti přetypovávání Node na Invalid/ValidNode (1 hodina)
			 \item GraphSVGImageGenerator (3 hodiny)
     \end{itemize}

    \item Jindřich Máca
      \begin{itemize}
      %vyplní Tuník 
       \item Administrace Redminu (1.6 hodiny)
			 \item Dostylování prezentační vrstvy (0.5 hodiny)
			 \item Oprava starší dokumentace podle nových programátorských konvencí (4 hodina)
			 \item Testovací webová stránka (0.5 hodiny)
			 \item FAQ pro Instalační příručku (2 hodiny)
     \end{itemize}

    \item Petr Svoboda
      \begin{itemize}
       \item Redesign WholePresentationController (2 hodiny)
       \item Analýza a testování nástrojů pro aplikaci CSS na DOM (5.5 hodiny)
			 \item Vytvoření testovací stránky a provedení testu CSSBox (1.5 hodiny)
       \item Implementace CSSRedundancyChecker (10.30 hodin)
     \end{itemize}
   \end{itemize}  
\end{frame} 

\begin{frame}[allowframebreaks]\frametitle{Důležité fragmenty naší práce}
  \begin{itemize}
    %vylpníme dohromady
    \item WholePresentationController
		\item CSSRedundancyChecker
		\item Instalační příručka
  \end{itemize}
\end{frame}

\begin{frame}[allowframebreaks]\frametitle{Na co jsme hrdí} 
  \begin{itemize}
    \item Jakub Koza
      \begin{itemize}
       %vyplní Kozík
       \item Instalační příručka
     \end{itemize}
   
    \item Adam Kugler
      \begin{itemize}
      %vyplní Adam
       \item SVG graf je skoro hotov, tím pádem máme téměř hotovou další iteraci a vypadá opravdu dobře.
     \end{itemize}

    \item Jindřich Máca
      \begin{itemize}
      %vyplní Tuník 
       \item Jsem rád, že se nám podařilo vyřešit CSSRedundancyChecker, protože ze začátku jsme si jím nebyli moc jistí.
      \end{itemize}  
   
    \item Petr Svoboda
      \begin{itemize}
      %vyplní Petr
       \item Jsem rád, že jsme použili neocenitelnou knihovnu CSSBox od kolegů z Brna.
     \end{itemize}
   \end{itemize}  
\end{frame}

\begin{frame}[allowframebreaks]\frametitle{Co jak šlo}
  \begin{itemize}
    \item Jakub Koza
     %vyplní Kozík
     \begin{block}{Špatně} %co nešlo
       \begin{itemize}
        \item Hledání důvodu chybného merge
       \end{itemize}
     \end{block}
     \begin{block}{Průměrně} %co šlo
        \begin{itemize}
        \item Instalační příručka
       \end{itemize}
     \end{block}
     \begin{block}{Dobře} %co šlo dobře
       \begin{itemize}
        \item Implementace WPC a Kernelu
       \end{itemize}
     \end{block}
   
    \item Adam Kugler
      %vyplní Adam
      \begin{block}{Špatně} %co nešlo
       \begin{itemize}
        \item CSSRedundancyChecker
       \end{itemize}
     \end{block}
     \begin{block}{Průměrně} %co šlo
        \begin{itemize}
         \item SVG graf
       \end{itemize}
     \end{block}
     \begin{block}{Dobře} %co šlo dobře
       \begin{itemize}
        \item Refaktoring
       \end{itemize}
     \end{block}
  
    \item Jindřich Máca
      %vyplní Tuník 
      \begin{block}{Špatně} %co nešlo
       \begin{itemize}
        \item Nedělal jsem nic, co by bylo těžké
       \end{itemize}
     \end{block}
     \begin{block}{Průměrně} %co šlo
        \begin{itemize}
         \item Oprava dokumentace podle nových programátorských konvencí
       \end{itemize}
     \end{block}
     \begin{block}{Dobře} %co šlo dobře
       \begin{itemize}
        \item Testovací webová stránka
       \end{itemize}
     \end{block}
   
    \item Petr Svoboda
      %vyplní Petr
      \begin{block}{Špatně} %co nešlo
       \begin{itemize}
        \item Objevování skryté chyby v CSSBoxu
       \end{itemize}
     \end{block}
     \begin{block}{Průměrně} %co šlo
        \begin{itemize}
        \item Implementace redundancy checkeru
       \end{itemize}
     \end{block}
     \begin{block}{Dobře} %co šlo dobře
       \begin{itemize}
        \item Testování externích knihoven
       \end{itemize}
     \end{block}
   \end{itemize}
\end{frame}

\begin{frame}[allowframebreaks]\frametitle{Plán do příští iterace}
  \begin{itemize}
    %vylpní vedoucí podle harmonogramu + se mohou věci přidat
		\item Grafická prezentace TraversalGraph pomocí Graphvizu
		\item GraphGenerator
		\item Presentation
		\item Uživatelská příručka
  \end{itemize}
\end{frame}

\begin{frame}[allowframebreaks]\frametitle{Přerozdělení bodů}
    
    \begin{center}
  \begin{tabular}{| c | c | c | c |}
    \hline
     J.~Koza & A.~Kugler & J.~Máca & P.~Svoboda \\
    \hline
    %vylpní vedoucí
     2 & 0 & -4 & 2 \\
    \hline
  \end{tabular}     
   \end{center}
\end{frame}

\end{document}